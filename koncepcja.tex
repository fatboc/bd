\documentclass[10pt,a4paper,notitlepage]{article}
\usepackage{fullpage}
\usepackage[utf8]{inputenc}
\usepackage[T1]{fontenc}
\usepackage[polish]{babel}
\title{Dziennik elektroniczny}
\author{Fatoumata Bocar, Radosław Błażewicz}
\date{}
\begin{document}
\maketitle
\section{Wstęp}
Celem projektu jest stworzenie bazy danych o funkcjonalności dziennika elektronicznego. Ma ona umożliwiać zarządzanie ocenami uczniów z poziomu nauczyciela oraz ich oglądanie z punktu widzenia ucznia oraz rodzica. Dla różnych grup użytkowników dostępne są różne operacje, w szczególności wyłącznie kadra pedagogiczna ma możliwość zmiany bądź wprowadzenia nowych danych.
Do stworzonej bazy danych dołączony będzie obsługiwany przez przeglądarkę interfejs graficzny umożliwiający jej obsługę.

\section{Wydajność}
Baza danych ma być zdolna do przetworzenia rekordów w minimalnie następujących liczbach:
\begin{itemize}


\item 6 klas
\item 8 przedmiotów
\item 30 uczniów na klasę - 180 uczniów
\item 20 ocen z przedmiotu  na ucznia - 2880 ocen
\item obecności wszystkich uczniów w ciągu roku (200 na ucznia) - 36000 obecności
\item pozostałe rekordy szacowane na 5000
\end{itemize}
daje łącznie możliwość przetworzenia ok. 45 000 rekordów. 
\section{Przetwarzane grupy danych}
\begin{itemize}
	\item Uczniowie (i rodzice)
	\item Nauczyciele
	\item Przedmioty
	\item Oceny
	\item Obecności
	\item Klasy
	\item Osoby
	\item Kontakty
	\item Rodzaje ocen
	\item Rodzaje kontaktów
	\item Opiekunowie
	\item Uwagi
	\item Obecności

\end{itemize}

\section{Funkcjonalności}
\begin{itemize}
\item Nauczyciel:
\begin{itemize}
\item Dodawanie, usuwanie i modyfikacja ocen ze swojego przedmiotu
\item Wstawianie uwag
\item Wyświetlenie ocen ze swojego przedmiotu
\item Dodawanie obecności
\end{itemize}
\item Wychowawca:
\begin{itemize}
\item Dodawanie, usuwanie i modyfikacja ocen z zachowania swojej klasy
\item Wstawianie uwag
\item Wyświetlenie ocen swojej klasy
\item Modyfikacja danych osobowych i kontaktowych swoich uczniów
\item Dodawanie obecności
\item Edycja danych ucznia
\end{itemize}


\item Uczeń i rodzic:
\begin{itemize}
\item Wyświetlenie swoich ocen
\item Wyświetlenie swoich uwag
\end{itemize}


\item Administrator
\begin{itemize}
\item Możliwość modyfikacji wszystkich rekordów
\item Modyfikacja praw dostępu
\end{itemize}
\end{itemize}


\section{Wymagania niefunkcjonalne}
\subsection{Skalowalnosć}
\begin{itemize}
\item Projekt będzie tworzony w sposób umożliwiający rozbudowanie bazy danych
\item Powinien działać na zestawach danych porównywalnych ze zbiorem testowym - np. szkoła 300 osób + 40 nauczycieli
\item Możliwość rozbudowy o dodatkowe elementy i funkcjonalności
\end{itemize}

\subsection{Bezpieczeństwo}
Ze względu na bezpieczeństwo bazy danych szczególna uwaga zostanie przywiązana do następujących kwestii:
\begin{itemize}
\item Przewidywanie podatności na etapie tworzenia
\item Zabezpieczenia uwierzytelniania
\item Logowanie prób uwierzytelnienia
\item Logowanie działań zabronionych dla użytkownika
\item Zapewnienie wydajności na stałym poziomie
\item Zapewnienie bezawaryjnego działania
\item Zapewnienie uprawnionym użytkownikom dostępu do odpowiednich grup rekordów
\item Opracowanie strategii kopii zapasowych
\end{itemize}

\end{document}